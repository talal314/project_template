\section{Conclusiones}
Finalmente, el objetivo que teníamos al inicio de la investigación de encontrar fármacos que puedan actuar sobre el virus del SARS-CoV-2 usando la base de datos ChEMBL ha sido completada. Al inicio del trabajo creamos el interactoma de las proteínas, pero debido a que hay muchos módulos dentro de ese mismo interactoma, es difícil encontrar un interactoma funcional completo. Otro dato, es que al hacer el mapeo solo se pierden la mitad casi de las proteínas, es decir, tenemos un alto nivel de interacción entre bastantes proteínas secundarias, por lo que será de mayor ayuda en la búsqueda de los fármacos, ya que saldrán resultados más certeros. Aun así, nos sale de resultados un gran número de fármacos ya aprobados. Además, dentro de estos fármacos podemos observar, que se encuentran un mecanismo de acción mayoritario, es el caso de fármacos inhibidoras. Como consecuencia de todo podemos concluir que hay que estudiar qué proteínas del interactoma funcional son las mejores candidatas que desarrollar un fármaco.  
